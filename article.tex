\documentclass{article}
\usepackage{magazine-style}

\begin{document}

\title{Scikit-learn: machine learning without learning the machinery}

\author{The authors}

\maketitle

\section{A software project across communities}

\paragraph{Project vision}
%
Bridging the gap between machine learning research and applications

\cite{pedregosa2011}

Impact on education (documentation)

Striving for ease-of-use and quality

Importance of API design \cite{buitinck2013ecml}

\paragraph{The python data ecosystem}
%
In a data analysis pipeline, machine learning is only a tiny bit of the
pipeline, although it is critical.

Great scientific and numeric ecosystem
\cite{oliphant2007python,varoquaux2013scipy}, but also text processing,
webservers

Numpy arrays \cite{vanderwalt2011}, Scipy, matplotlib. Cython
\cite{behnel2011cython}

Pandas for columnar data, scikit-image, NLTK.

Ipython for interactive work \cite{perez2007ipython}

\paragraph{Some history of the project}
%
PhD project of David Cournapeau and Matthieu Brucher, Parietal-INRIA
comes along,... 2010 sprint, 

LibSVM a good start
\cite{chang2011libsvm}

\section{A brief introduction to machine learning}

Discuss what we can learn from the wages examples. Introduce concepts of
overfitting, model complexity

\section{Learning with scikit-learn}

\paragraph{The data matrix}
%
Blahblah

\paragraph{Supervised models: learning to predict}
%
Blahblah

\paragraph{Model evaluation and parameter selection}
%
Model evaluation:
cross-validation: the concept, cross-val iterators, and cross\_val\_score

Problem of multiple metrics.

CV objects: GridSearchCV and RandomSearchCV, as well as FooBarCV objects.

\paragraph{Unsupervised models: learning to transform}
%
Unsupervised learning: a variety of usage pattern.

Many things can be seen as Transformers

Introduce the pipeline

\section{In practice}

\paragraph{A simple text-mining example}
%
Example from Lars here

\paragraph{What about big data?}
%
Partial fit, random projections, HashingVectorizer

\section{Nurturing an open source project}

Goal: enable anybody to contribute, have a controlled process, grow.

Github, tests (link to travis)

Difficulty of getting credit, of rewarding properly the long tail of small
contributors, of finding funding. Problem of brain drain.

Difficulty of project scope (ideally: first cover all of statistical
learning classics \cite{elemstatlearn}) and default parameters

We have found that implementing, even a standard algorithm, really well,
can require a lot of domain knowledge. Thus it is natural that specific
libraries span up to solve it. Some adopt scikit-learn API and standard,
and we hope that scikit-learn has a structural effect on the environment.

\small
\bibliography{paper}
\bibliographystyle{plain}


\end{document}
